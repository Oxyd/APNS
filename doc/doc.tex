\documentclass{article}
\usepackage[czech]{babel}
\usepackage[utf8]{inputenc}
\usepackage[T1]{fontenc}
\usepackage{url}
\usepackage{fullpage}

\begin{document}
\begin{flushright}
  Ondřej Majerech
\end{flushright}
\begin{center}
  \Huge Dokumentace ročníkového projektu
\end{center}

\section{Účel}
Program je primárně experimentální implementací algoritmu \emph{Proof-Number Search} v adaptaci pro hru \emph{Arimaa}.
Má ukázat vhodnost zmíněného algoritmu při řešení koncových částí hry. Program má spíše teoretický význam, protože jeho
hlavním výstupem je částečně prohledaný stavový prostor hry -- tento výstup je ale možné použít jako vstup jiného
programu.

\section{Podporované systémy}
Program je určený pro systémy Microsoft Windows a FreeBSD. K běhu je na obou systémech potřeba mít instalovaný
Python verze 2.7 (pro Windows ke stažení na \url{http://python.org/download/}). 

Pro Windows je připraven zip soubor obsahující vše potřebné pro spuštění programu -- ten lze získat na adrese
\url{https://github.com/Oxyd/APNS} kliknutím na tlačítko \emph{Download} a následně stažením souboru
\texttt{apns-windows-32bit.zip} pro 32bitové Windows či \texttt{apns-windows-64bit.zip} pro 64bitový systém.

 Na platformě FreeBSD je potřeba prvně program přeložit -- o tom více v části \ref{sec:compiling}.

\section{Běh programu}
\label{sec:running}
Program je možné spustit buďto s grafickým rozhraním či v dávkovém režimu. Grafické rozhraní zobrazuje spočtený strom a
v každém jeho vrcholu zobrazuje příslušnou situaci na hrací ploše. 

\subsection{Grafické rozhraní}
Grafické rozhraní lze ve Windows spustit poklepáním na ikonu souboru \texttt{gui.pyw}. Ve Windows i FreeBSD lze grafické
rozhraní spustit příkazem \texttt{python gui.pyw} z příkazové řádky v adresáři s programem.

Grafické rozhraní v levé části zobrazuje výsledný strom, pravá část ukazuje příslušnou herní pozici po vybrání některého
z vrcholů stromu. Tlačítka v horní části rozhraní umožňují výběr nové počáteční pozice, zahájení výpočtu, načtení
předchozího výpočtu z disku, uložení výpočtu na disk a uložení vybrané pozice na disk.

Po skončení výpočtu je možné procházet výsledný strom, který je zobrazený v levé části okna programu. Sloupec
\emph{Step} určuje krok vedoucí do příslušného vrcholu; \emph{Type} říká, jakého typu je příslušný vrchol -- buď
\emph{OR} (tedy tento vrchol vznikl z rodičovského tahem původního hráče) nebo \emph{AND}; \emph{PN} a \emph{DN}
zobrazují hodnoty \emph{proof number} a \emph{disproof number} pro zvolený vrchol.

Vybráním některého vrcholu se v pravé části zobrazí pozice odpovídající stavu hry po provedení kroku v sloupci
\emph{Step}.

Nulová hodnota \emph{PN} znamená, že příslušný uzel stromu je \emph{dokázaný} -- tedy pokud se hra dostane do stavu
představovaného tímto vrcholem, pak začínající hráč může zaručeně vyhrát. Podobně, je-li hodnota \emph{DN} nulová,
znamená to, že dostane-li se hra do tohoto stavu, nemůže původní hráč vyhrát, protože tato pozice zaručuje vítězství
protivníkovi. Takový vrchol se nazývá \emph{vyvrácený}.

\subsubsection{Příklad 1: Prozkoumání stavového prostoru vedoucího z jednoduché pozice}
V souboru \texttt{puzzle1-one-move.txt} v adresáři \texttt{example-positions} je připravena jednoduchá startovní pozice.
Řekněme, že chceme prozkoumat všechny možnosti, abychom zjistili, zda a jak může zlatý hráč zvítězit.

Klepnutím na \emph{New Initial Position} se zobrazí dialogové okno umožňující zadání nové pozice či její načtení z
disku. Jelikož zmíněná pozice je již připravena v souboru, stačí z adreséře \texttt{example-positions} vybrat soubor
\texttt{puzzle1-{}one-{}move.txt}. Načtená pozice se pak zobrazí na obrazovce a je možné ji dále upravovat. Stiskem
\texttt{Ok} se pozice načte jako kořenová pozice stromu.

Nyní je možné spustit výpočet. Po stisku \emph{Run Search} se zobrazí dialog umožňující nastavit dobu trvání výpočtu
a velikost použité transpoziční tabulky. Pro zmíněnou pozici, která je velmi jednoduchá, jsou výchozí parametry naprosto
postačující. Stiskem \emph{Run} se zahájí výpočet.

Hodnota 0 v sloupci \emph{PN} znamená, že příslušný vrchol je \emph{dokázaný}. V tomto konkrétním příkladě je kořenový
vrchol dokázaný -- tedy zlatý hráč může z původní pozice zaručeně zvítězit. Vítězství je zaručené, protože zlatý hráč
může zvolit krok, který zaručuje vítězství -- v našem případě je to krok \emph{Dc7n} neboli přesun psa z pozice c7
směrem vzhůru, který má hodnotu \emph{PN} rovněž rovnou nule.

Po rozevření vrcholu \emph{Dc7n} je vidět, že z něj je možné provést několik dalších kroků, včetně ukončení tahu a
přenechání hry soupeři (to jsou vrcholy typu \emph{AND}). Opět je zde vrchol -- \emph{Dc8s rd8w}, neboli pes z c8 se
přesune dolů a potáhne za sebou zajíce z d8 --, který zaručuje vítězství, protože jeho hodnota \emph{PN} je nulová. Po
rozbalení tohoto vrcholu se zobrazí další možné tahy. Mezi nimi je \emph{Rd7n}, kterým se zlatý zajíc z d7 přesune do
cílové pozice -- tedy zlatý hráč zvítězí.

Celkově je tedy vidět, že ze startovní pozice může zlatý hráč zvítězit volbou kroků \emph{Dc7n Dc8s rd8w Rd7n}.

Za zmínku též stojí fakt, že jedním z potomků vrcholu \emph{Dc8s rd8w} je vrchol \emph{Rd7n dd6n}, který má též hodnotu
\emph{PN} rovnu nule. Tento vrchol je ovšem typu \emph{AND}, neboli představuje situaci, kdy se zlatý hráč rozhodl svůj
tah ukončit po kroku \emph{Dc8s rd8w} a stříbrný jako svůj první krok zvolil \emph{Dc8s rd8w}, což je odtlačení zlatého
zajíce do cílové pozice. Nedá se však očekávat, že by stříbrný hráč tento krok zvolil vždy (má i jiné možnosti, které
program neprozkoumal), a tedy tento vrchol samotný nedokazuje, že zlatý hráč nutně zvítězí.

\subsubsection{Příklad 2: Načítání a ukládání stavového prostoru}
\label{sec:gui-example-2}
Pro tento příklad zvolme jako počáteční pozici soubor \texttt{gold-to-play-and-win-in-2-moves.txt}. Opět jako v příkladu
1 stačí zvolit \emph{New Initial Position}, \emph{Load} a zvolit příslušný soubor v adresáři s příklady. Tato pozice je
však na výpočet daleko složitější než ta z prvního příkladu.

Volbou \emph{Run Search} opět spustíme výpočet -- díky složitosti této pozice se dá očekávat, že na běžném PC program
nevydá definitivní výsledek ani po přednastaveném šedesátisekundovém limitu.

Je-li tedy již strom (čátečně) spočítán, můžeme zvolit \emph{Save Search}, vybrat jméno souboru, kam se strom uloží --
například \texttt{search-1.txt} -- a potvrdit tlačítkem \emph{Ok}. Pro demonstrační účely je nyní možné program ukončit
a poté opět spustit. Pak stiskem \emph{Load Search} znovu vybrat soubor \texttt{search-1.txt}. Program by nyní měl
zobrazovat stejný strom jako před ukončením.

Dále je možné třeba znovu stisknout \emph{Run Search} a pokračovat ve výpočtu.

\subsection{Dávkové rozhraní}
Dávkové rozhraní umožňuje provádění výpočtu z příkazové řádky či dávkového souboru bez nutnosti použití grafického
rozhraní. Všechny parametry výpočtu musí být zadány jako parametry na příkazové řádce. Výpočet je kdykoliv možné ukončit
stiskem kláves \texttt{Ctrl-C}.

\subsubsection{Spouštění a parametry příkazové řádky}
Z příkazové řádky je možné dávkové rozhraní programu spustit příkazem \texttt{python batch.py \textit{parametry}}.
Nejsou-li zadány žádné parametry, vypíše program stručný soupis všech parametrů a skončí. Stručnou nápovědu je možné
vyvolat parametrem \texttt{--help}, tedy příkazem \texttt{python batch.py --help}.

Důležité parametry jsou:\begin{description}
  \item[\texttt{-p} \textit{pozice}] Udává cestu k souboru s počáteční pozicí. Jeho funkce je analogická funkci
  \emph{New Initial Position} z grafického rozhraní.
  \item[\texttt{-d} \textit{výstup}] Udává cestu k souboru, do kterého se uloží spočtený strom. Tato funkce je analogem
  tlačítka \emph{Save Search} z grafického rozhraní.
  \item[\texttt{-s} \textit{strom}] Udává cestu k souboru, ve kterém je částečně spočtený vyhledávací strom, ve kterém
  se má pokračovat ve výpočtu. Tedy stejná funkce, kterou v grafickém rozhraní zastává tlačítko \emph{Load Search}.
  \item[\texttt{-t} \textit{sekundy}] Umožňuje nastavit časový limit (v sekundách) pro vyhledávání (výchozí hodnota je
  60 sekund)
\end{description}

Parametr \texttt{-d} je povinný. Dále uživatel musí zadat buď parametr \texttt{-p} nebo parametr \texttt{-s}.

\subsubsection{Příklad 1: Prozkoumávání stavového stromu z dané pozice}
Opět jako v prvním příkladu u grafického rozhraní budeme chtít spočítat příslušný strom, máme-li v souboru uloženou
počáteční pozici. Pro tento příklad zvolme třeba pozici \texttt{puzzle3-one-move.txt}, pro kterou program dokáže vydat
kladný výsledek (zlatý hráč vítězí) na mém PC po přibližně 7 sekundách. Řekněme, že výsledek chceme uložit do souboru
\texttt{search-2.txt}.

Příslušný příkaz včetně parametrů tedy bude vypadat takto:
\begin{center}\verb+python batch.py -p example-positions/puzzle3-one-move.txt -d search-2.txt+\end{center}

Pokud je výpočet přerušen ještě před jeho dokončením, program stejně vypíše částečně spočtený strom do specifikovaného
souboru.

Výstup, tedy soubor \texttt{search-2.txt}, je pak možné otevřít v grafickém rozhraní a procházet jak bylo uvedeno v
předchozích částech.

\subsubsection{Příklad 2: Pokračování v předešlém výpočtu}
Řekněme, že chceme pokračovat ve výpočtu, který jsme začali v části \ref{sec:gui-example-2} -- tedy máme soubor
\texttt{search-1.txt}, který je částečným výsledekm prohledávání z pozice \texttt{gold-to-play-and-win-in-2-moves.txt}.
Výsledek chceme uložit do souboru \texttt{search-3.txt}.

Nyní tedy místo parametru \texttt{-p} uvedeme parametr \texttt{-s} specifikující předešlé vyhledávání. Tedy -- za
předpokladu, že soubor \texttt{search-1.txt} je uložen v kořenovém adresáři programu -- bude celý příkaz vypadat
následovně:
\begin{center}\verb+python batch.py -s search-1.txt -d search-3.txt+\end{center}

\subsection{Transpoziční tabulka a její vliv na výpočet}
Během výpočtu si program pamatuje některé navštívené pozice v \emph{transpoziční tabulce}. To umožňuje zrychlení
výpočtu, neboť někdy je možné ke stejné pozici dojít vícero způsoby a transpoziční tabulka pak zajišťuje, že není třeba
stejnou pozici vyhodnocovat vícekrát.

Do transpoziční tabulky se však nemohou vejít všechny záznamy. Počet záznamů, které se do tabulky vejdou, je přímo určen
její velikostí v paměti. Velikost transpoziční tabulky lze nastavit v grafickém rozhraní volbou \emph{Transposition
Table Size}, v řádkovém rozhraní pak parametrem \texttt{-r \textit{velikost v MB}}.

\subsection{Zobrazované informace během výpočtu}
Během prohledávání stavového prostoru program vypisuje některé informace. Mezi ně patří \begin{description}
\item[\emph{Search Memory Usage}] Přibližné množství paměti používané programem. Tato hodnota nezahrnuje všech\-nu
použitou paměť, a tedy zobrazená hodnota bude vždy nižší než hodnota skutečná. Přesto však může sloužit jako dobrý
ukazatel paměťové náročnosti programu.
\item[\emph{Unique Positions Total}] Celkový počet unikátních pozic. Tento údaj nezohledňuje případné několikanásobné
použití stejné pozice vícekrát díky transpoziční tabulce. Může se tedy stát, že výsledný strom má pak zdánlivě více
vrcholů než je tato hodnota.
\item[\emph{New Positions per Second}] Navýšení hodnoty \emph{Unique Positions Total} za poslední sekundu.
\item[\emph{Transposition Table Size}] Paměť použitá transpoziční tabulkou. Tato hodnota není započítána do \emph{Search
Memory Usage}.
\item[\emph{Transposition Table Hits}] (Pouze v grafickém rozhraní) Počet úspěšných vyhledání v transpoziční tabulce.
\item[\emph{Transposition Table Misses}] (Pouze v grafickém rozhraní) Počet neúspěšných pokusů o vyhledání v
transpoziční tabulce.
\end{description}

\section{Překlad programu}
\label{sec:compiling}

\subsection{Prerekvizity}
Pro překlad je třeba mít kompletní zdrojový kód programu. Ten lze obstarat přes systém git příkazem
\begin{center}\verb+git clone git://github.com/Oxyd/APNS.git+\end{center} či stažením zdrojového archivu ve formátu
\texttt{.zip} či \texttt{.tar.gz} z \url{https://github.com/Oxyd/APNS}.

Dále je třeba mít v systému instalovány následující knihovny a nástroje\begin{description}
\item[SCons] Používaný nástroj pro sestavení programu. Instalátor pro Windows je možné získat na
\url{http://scons.org/}. Na systému FreeBSD stačí instalovat port \texttt{devel/scons}.
\item[Boost] Kolekce knihoven, z nichž některé program používá. Je třeba mít verzi aspoň 1.44. Stáhnout lze z
\url{http://boost.org/}, kde jsou i informace pro sestavení kolekce. Na FreeBSD je možné instalovat port
\texttt{devel/boost-all}.
\item[Google Test] Není nezbytně nutný pro překlad; používá se pouze pro jednotkové testy. Na FreeBSD existuje v portu
\texttt{devel/googletest}. Pro Windows se stačí obrátit na stránky projektu na
\url{http://code.google.com/p/googletest/}.
\end{description}

Je třeba dále mít k dispozici překladač jazyka C++. Sestavovací skript automaticky použije překladač z \emph{Microsoft
Visual Studia} na systému Windows -- je třeba mít verzi 10. (Měla by stačit i verze 9 a 8, ale to není otestované.) Na
FreeBSD se pro překlad používá překladač \emph{GCC} -- je ovšem třeba mít překladač verze aspoň 4.5. Ten lze instalovat
z portu \texttt{lang/gcc45}.

Překlad na Windows by měl být možný i pomocí kolekce \emph{MinGW} -- sestavovací skript to však nepodporuje, a bylo by
tedy nutné ho před tím upravit. Dále očekávám, že program půjde beze změn či jenom s minimálními změnami přeložit a
spustit i na dalších systémech, zejména různých distribucích systému \emph{GNU/Linux} -- to jsem ovšem netestoval, a
výsledek není zaručen. Obdobně očekávám možnost překladu programu překladačem \emph{Clang} -- ani to jsem však
netestoval.

\subsection{Konfigurace pro překlad}
Než je možné program přeložit, je třeba určit umístění používaných knihoven na systému, kde má překlad probíhat. To se
provádí úpravou souboru \texttt{build\_config.py}. V souboru jsou komentáře vysvětlující jak nastavit příslušné hodnoty.

\subsection{Překlad}
Překlad lze spustit příkazem \verb+scons+ v kořenovém adresáři se zdrojovým kódem programu. Pokud překlad proběhne
úspěšně, není třeba žádných dalších kroků; program je spustitelný tak, jak bylo popsáno v části \ref{sec:running}.

Pro účely vývoje programu může být vhodné přeložit program tak, aby obsahoval ladicí informace. V tom případě je třeba
překlad spustit příkazem \verb+scons debug=1+. Navíc je možné specifikovat cíl \texttt{tests}, který přeloží a provede
jednotkové testy -- tedy použít příkaz \verb+scons tests+. Jednotlivé jednotkové testy je možné vyvolat cílem
\emph{jméno\_testu}, kde \emph{jméno\_testu} odpovídá jménu souboru v adresáři \texttt{apns\_module/tests} bez koncovky
\texttt{.cpp}. Například tedy pro spuštění jednotkových testů ze souboru \texttt{apns\_module/tests/board\_test.cpp} se
dá použít příkaz \verb+scons board_test+.

Při překladu na systému Windows je též možné specifikovat parametr \texttt{bits=32} nebo \texttt{bits=64}, který určuje,
zda má výsledný program být přeložen pro 32bitovou nebo 64bitovou verzi systému. Kromě toho existuje cíl
\texttt{distrib}, který vytvoří \texttt{zip} soubor s binární distribucí programu.

\section{Vstupní a výstupní data}
\subsection{Pozice}
Vstupní pozice je textový soubor ve formátu popsaném na adrese \url{http://arimaa.com/arimaa/learn/notation.html} v
části \emph{Notation for Recording Arimaa Positions}. Program ale ignoruje číslo tahu a předešlé kroky.

\subsection{Strom}
Stavový strom je ukládán v textovém formátu. Struktura souboru je \begin{itemize}
  \item Na první řádce je počáteční pozice v \emph{kompaktním formátu}
  \item Na druhé řádce je \begin{itemize}
    \item Hráč, který je první na tahu jako řetězec buď \texttt{gold} nebo \texttt{silver}.
    \item Počet vrcholů stromu v desítkové soustavě
    \item Obě hodnoty jsou odděleny mezerou
  \end{itemize}
  Například tedy: \begin{center}\verb+gold 100617+\end{center}
  \item Následuje seznam vrcholů stromu. Jeden vrchol je na jedné řádce. Každý vrchol je popsán
  následovně: \begin{center}\texttt{\textit{id}: \textit{tah}: \textit{typ} \textit{kroky} \textit{PN}
  \textit{DN}}\end{center}
  Kde jednotlivé položky jsou:\begin{description}
  \item[\textit{id}] Přirozené číslo v dekadickém zápisu sloužící jako jednoznačný identifikátor vrcholu. Kořen má vždy
  \textit{id} rovno 1.
  \item[\textit{tah}] Textový pozic tahu vedoucího do této pozice. Používá se oficiální notace pro zápis tahů hry
  Arimaa. Kořen má místo tahu řetězec \texttt{root}.
  \item[\textit{typ}] Řetězec \texttt{or} nebo řetězec \texttt{and}. Udává typ vrcholu.
  \item[\textit{kroky}] Číslo v dekadickém zápisu udávající počet kroků, které ještě může hráč učinit v rámci svého
  tahu.
  \item[\textit{PN}] Hodnota \emph{proof number} pro tento vrchol v dekadickém zápisu nebo řetězec ,,\textit{infty}``
  značící nekonečno.
  \item[\textit{DN}] Hodnota \emph{disproof number} pro tento vrchol v dekadickém zápisu nebo řetězec ,,\textit{infty}``
  značící nekonečno.
  \end{description}
  \item Následně řetězec ,,\texttt{---}`` sloužící jako pomocný oddělovač mezi seznamem vrcholů a seznamem hran.
  \item Seznam hran. Každá hrana je na jedné řádce. Hrana je popsána řádkou
  	\begin{center}\textit{id}\texttt{ : }[\textit{syn }]*\end{center}
  	\begin{description}
  	\item[\textit{id}] Je identifikátor vrcholu. Tento identifikátor musí odkazovat na některý vrchol ze seznamu vrcholů.
  	\item[\textit{syn}] Opět platný identifikátor vrcholu. 
  	\end{description}
  	Jedna taková řádka znamená, že vrchol \textit{id} má za syny vrcholy uvedené za dvojtečkou. Jedna hrana může vypadat
  	\begin{center}\verb+4 : 7 6 5+\end{center}Neboli, vrchol 4 má za syny vrcholy 5, 6 a 7.
  	
  	Pokud vrchol nemá žádné syny (je to list), pak nemusí být v seznamu uveden, případně může být uveden s prázdným
  	seznamem za dvojtečkou: \verb+8 : +.
\end{itemize}
\end{document}
% vim:textwidth=120
