\documentclass{article}
\usepackage[czech]{babel}
\usepackage[utf8]{inputenc}
\usepackage[T1]{fontenc}
\usepackage{url}
\usepackage{fullpage}
\usepackage{algorithmic}
\usepackage{algorithm}
\usepackage{amsmath}
\usepackage{float}

\begin{document}
\begin{flushright}
  Ondřej Majerech
\end{flushright}
\begin{center}
  \Huge Dokumentace ročníkového projektu
\end{center}

\section{Účel}
Program je primárně experimentální implementací algoritmu \emph{Proof-Number
Search} v adaptaci pro hru \emph{Arimaa}.
Má ukázat vhodnost zmíněného algoritmu při řešení koncových částí hry. Program
má spíše teoretický význam, protože jeho
hlavním výstupem je částečně prohledaný stavový prostor hry -- tento výstup je
ale možné použít jako vstup jiného
programu.

\section{Podporované systémy}
Program je určený pro systémy Microsoft Windows a GNU/Linux. K běhu je na obou
systémech potřeba mít instalovaný
Python verze 2.7 (pro Windows ke stažení na \url{http://python.org/download/}). 

Pro Windows je připraven zip soubor obsahující vše potřebné pro spuštění
programu -- ten lze získat na adrese
\url{https://github.com/Oxyd/APNS} kliknutím na tlačítko \emph{Download} a
následně stažením souboru
\texttt{apns-windows-32bit-\textit{verze}.zip} pro 32bitové Windows či
\texttt{apns-windows-64bit-\textit{verze}.zip} pro 
64bitový systém.

Na platformě GNU/Linux je potřeba prvně program přeložit -- o tom více v části
\ref{sec:compiling}.

\section{Běh programu}
\label{sec:running}
Program je možné spustit buďto s grafickým rozhraním či v dávkovém režimu.
Grafické rozhraní zobrazuje spočtený strom a v každém jeho vrcholu zobrazuje 
příslušnou situaci na hrací ploše. 

\subsection{Grafické rozhraní}
Grafické rozhraní lze ve Windows spustit poklepáním na ikonu souboru
\texttt{gui.pyw}. Ve Windows i GNU/Linuxu lze grafické rozhraní spustit příkazem 
\texttt{python gui.pyw} z příkazové řádky v adresáři s programem.

Grafické rozhraní v levé části zobrazuje výsledný strom, pravá část ukazuje
příslušnou herní pozici po vybrání některého z vrcholů stromu. Tlačítka v horní 
části rozhraní umožňují výběr nové počáteční pozice, zahájení výpočtu, zahození
dosavadního výpočtu, načtení předem provedeného výpočtu z disku, uložení výpočtu 
na disk, uložení vybrané pozice na disk či zobrazení statistik o posledním 
provedeném výpočtu.

Po skončení či přerušení výpočtu je možné procházet výsledný strom, který je 
zobrazený v levé části okna programu. Sloupec \emph{Step} určuje krok vedoucí do 
příslušného vrcholu; \emph{Type} říká, jakého typu je příslušný vrchol -- buď
\emph{OR} pro vrchol, ze kterého táhne útočník, či \emph{AND}, což je stav, ze
kterého táhne obránce.

Vybráním některého vrcholu se v pravé části zobrazí pozice odpovídající stavu
hry po provedení kroku v sloupci \emph{Step}.

Nulová hodnota \emph{PN} znamená, že příslušný uzel stromu je \emph{dokázaný} --
tedy pokud se hra dostane do stavu představovaného tímto vrcholem, pak útočník 
může zaručeně vyhrát. Podobně, je-li hodnota \emph{DN} nulová, znamená to, že 
dostane-li se hra do tohoto stavu, nemůže útočník vyhrát, protože tato pozice 
zaručuje vítězství obránci. Takový vrchol se nazývá \emph{vyvrácený}.

V každé úrovni stromu -- tedy v možných krocích z jedné pozice -- je vždy jeden
krok označen hvězdičkou -- ta označuje nejlepší možný tah (vzhledem k číslům 
\emph{PN} a \emph{DN}) na této úrovni. Dvěma hvězdičkami je pak označena 
principiální varianta.

\subsubsection{Příklad 1: Prozkoumání stavového prostoru vedoucího z jednoduché
pozice}
\paragraph{Spuštění výpočtu}
V souboru \texttt{puzzle1-one-move.txt} v adresáři \texttt{example-positions} je
připravena jednoduchá startovní pozice.
Řekněme, že chceme prozkoumat všechny možnosti, abychom zjistili, zda a jak může
zlatý hráč zvítězit.

Klepnutím na \emph{New Initial Position} se zobrazí dialogové okno umožňující
zadání nové pozice či její načtení z disku. Jelikož zmíněná pozice je již 
připravena v souboru, stačí z adreséře \texttt{example-positions} vybrat soubor
\texttt{puzzle1-{}one-{}move.txt}. Načtená pozice se pak zobrazí na obrazovce a
je možné ji dále upravovat. Stiskem \texttt{Ok} se pozice načte jako kořenová 
pozice stromu.

Nyní je možné spustit výpočet. Po stisku \emph{Run Search} se zobrazí dialog
umožňující nastavit různé parametry pro výpočet. Pro zmíněnou pozici, která je 
velmi jednoduchá, jsou výchozí parametry naprosto postačující. Stiskem 
\emph{Run} se zahájí výpočet.

\paragraph{Výsledky}
V tomto příkladě se velmi rychle povede dokázat kořen stromu -- hodnota 
\emph{PN} kořene je nulová. To znamená, že se programu povedlo dokázat, že zlatý 
hráč má zaručené vítězství. To je zaručené krokem \emph{Dc7n} -- tedy přesunem 
psa z C7 směrem vzhůru -- protože stav vzniklý tímto krokem je opět dokázaný. Po
kliknutí na \emph{Dc7n} se pravá část okna programu aktualizuje a zobrazí pozici 
vzniklou provedením tohoto kroku.

Z této pozice je možné pokračovat krokem \emph{Dc8s rd8w} -- odtažení stříbrného
zajíce psem doleva. Opět je to dokázaná pozice. Pod ní konečně je pozice 
\emph{Rd7n} -- posun zlatého zajíce do cílové řady, neboli vítězství zlatého 
hráče.

Celkově je tedy vidět, že ze startovní pozice může zlatý hráč zvítězit volbou
kroků \emph{Dc7n Dc8s rd8w Rd7n}.

První tři kroky v této posloupnosti jsou typu \emph{OR} -- jsou to kroky, ze
kterých dál táhne zlatý hráč. Poslední krok je typu \emph{AND}, protože by z něj 
táhl hráč stříbrný (pokud by neprohrál). Většina pozic se ve stromě vyskytuje
dvakrát -- pokaždé s jiným typem. Pozice typu \emph{OR} představuje situaci, kdy
zlatý hráč udělá krok a pokračuje ve svém tahu; \emph{AND} se stejným krokem 
znamená, že po provedení kroku zlatý hráč svůj tah skončí a nechá hrát soupeře.

\subsubsection{Příklad 2: Načítání a ukládání stavového prostoru}
\label{sec:gui-example-2}
Pro tento příklad zvolme jako počáteční pozici soubor
\texttt{puzzle6-two-moves.txt}. Opět jako v příkladu 1 stačí zvolit 
\emph{New Initial Position}, \emph{Load} a zvolit příslušný soubor v adresáři s 
příklady. Tato pozice je však na výpočet daleko složitější než ta z prvního 
příkladu.

Volbou \emph{Run Search} opět spustíme výpočet -- díky složitosti této pozice se
dá očekávat, že na běžném PC program nevydá definitivní výsledek ani po 
přednastaveném šedesátisekundovém limitu.

Je-li tedy již strom (čátečně) spočítán, můžeme zvolit \emph{Save Search},
vybrat jméno souboru, kam se strom uloží -- například \texttt{search-1.txt} -- 
a potvrdit tlačítkem \emph{Ok}. Pro demonstrační účely je nyní možné program 
ukončit a poté opět spustit. Pak stiskem \emph{Load Search} znovu vybrat soubor
\texttt{search-1.txt}. Program by nyní měl zobrazovat stejný strom jako před 
ukončením.

Dále je možné třeba znovu stisknout \emph{Run Search} a pokračovat ve výpočtu.

\subsection{Dávkové rozhraní}
Dávkové rozhraní umožňuje provádění výpočtu z příkazové řádky či dávkového
souboru bez nutnosti použití grafického rozhraní. Všechny parametry výpočtu musí 
být zadány jako parametry na příkazové řádce. Výpočet je kdykoliv možné ukončit
stiskem kláves \texttt{Ctrl-C}.

\subsubsection{Spouštění a parametry příkazové řádky}
Z příkazové řádky je možné dávkové rozhraní programu spustit příkazem
\texttt{python batch.py \textit{parametry}}.
Nejsou-li zadány žádné parametry, vypíše program stručný soupis všech parametrů
a skončí. Stručnou nápovědu je možné
vyvolat parametrem \texttt{--help}, tedy příkazem \texttt{python batch.py
--help}.

Důležité parametry jsou:\begin{description}
\item[\texttt{-p} \textit{pozice}] Udává cestu k souboru s počáteční pozicí.
  Jeho funkce je analogická funkci \emph{New Initial Position} z grafického 
  rozhraní.
\item[\texttt{-d} \textit{výstup}] Udává cestu k souboru, do kterého se uloží
  spočtený strom. Tato funkce je analogem tlačítka \emph{Save Search} z 
  grafického rozhraní.
\item[\texttt{-s} \textit{strom}] Udává cestu k souboru, ve kterém je částečně
  spočtený vyhledávací strom, ve kterém se má pokračovat ve výpočtu. Tedy stejná
  funkce, kterou v grafickém rozhraní zastává tlačítko \emph{Load Search}.
\item[\texttt{-t} \textit{sekundy}] Umožňuje nastavit časový limit (v sekundách) 
  pro vyhledávání (výchozí hodnota je 60 sekund)
\item[\texttt{-r} \textit{velikost v MB}] Nastavuje velikost použité transpoziční 
  tabulky. Nastavením na hodnotu 0 se zabrání jakémukoliv použití transpoziční 
  tabulky. Výchozí hodnota je 32 MB. 
\item[\texttt{-o} \textit{velikost v MB}] Podobně jako parametr \texttt{-r},
  akorát tento ovlivňuje důkazovou tabulku.
\item[\texttt{-m} \textit{velikost v MB}] Omezení maximální použité paměti při 
  výpočtu. Hodnota 0 znamená, že výpočet nemá být omezen velikostí použité
  paměti. Výchozí hodnota je na 64bitových systémech 0, na 32bitových systémech
  1500 MB.
\end{description}

Uživatel musí zadat parametr \texttt{-p} nebo \texttt{-s} (ne však oba zároveň).
Parametr \texttt{-d} je nepovinný, ale bez jeho určení se v dávkovém rozhraní
spočtený strom po dokončení práce zahodí.

\subsubsection{Příklad 1: Prozkoumávání stavového stromu z dané pozice}
Opět jako v prvním příkladu u grafického rozhraní budeme chtít spočítat 
příslušný strom, máme-li v souboru uloženou počáteční pozici. Pro tento příklad 
zvolme třeba pozici \texttt{puzzle3-one-move.txt}, pro kterou program dokáže 
vydat kladný výsledek (zlatý hráč vítězí) na mém PC\footnote{Procesor: AMD 
Athlon 64 X2 5400+ na 2,8 GHz, paměť: 3 GB DDR2 533, systém: Linux 3.3}  po 
přibližně jedné sekundě. Řekněme, že výsledek chceme uložit do souboru
\texttt{search-2.txt}.

Příslušný příkaz včetně parametrů tedy bude vypadat takto:
\begin{center}\texttt{python batch.py -p example-positions/puzzle3-one-move.txt 
-d search-2.txt}\end{center}

Pokud je výpočet přerušen ještě před jeho dokončením, program stejně vypíše
částečně spočtený strom do specifikovaného
souboru.

Výstup, tedy soubor \texttt{search-2.txt}, je pak možné otevřít v grafickém
rozhraní a procházet jak bylo uvedeno v předchozích částech.

\subsubsection{Příklad 2: Pokračování v předešlém výpočtu}
Řekněme, že chceme pokračovat ve výpočtu, který jsme začali v části
\ref{sec:gui-example-2} -- tedy máme soubor \texttt{search-1.txt}, který je 
částečným výsledekm prohledávání z pozice \texttt{puzzle6-two-moves.txt}.
Výsledek chceme uložit do souboru \texttt{search-3.txt}.

Nyní tedy místo parametru \texttt{-p} uvedeme parametr \texttt{-s} specifikující
předešlé vyhledávání. Tedy -- za předpokladu, že soubor \texttt{search-1.txt} je 
uložen v kořenovém adresáři programu -- bude celý příkaz vypadat následovně:
\begin{center}\texttt{python batch.py -s search-1.txt -d search-3.txt}
\end{center}

\subsection{Zobrazované informace během výpočtu}
Během prohledávání stavového prostoru program vypisuje některé informace. Mezi
ně patří \begin{description}
\item[\emph{Search Memory Usage}] Přibližné množství paměti používané programem.
  Tato hodnota nezahrnuje všechnu použitou paměť, a tedy zobrazená hodnota bude 
  vždy nižší než hodnota skutečná. Přesto však může sloužit jako dobrý ukazatel 
  paměťové náročnosti programu.
\item[\emph{Unique Positions Total}] Celkový počet vrcholů ve stromě.
\item[\emph{New Positions per Second}] Navýšení hodnoty \emph{Unique Positions
  Total} za poslední sekundu.
\item[\emph{Root PN} a \emph{Root DN}] Okamžité hodnoty \emph{PN} a \emph{DN}
  kořenového vrcholu.
\item[\emph{Transposition Table Size}] Paměť použitá transpoziční tabulkou.
\item[\emph{Proof Table Size}] Paměť použitá důkazovou tabulkou.
\item[\emph{Transposition Table Hits}] Počet úspěšných vyhledání v transpoziční
  tabulce.
\item[\emph{Transposition Table Misses}] Počet neúspěšných pokusů o vyhledání v
  transpoziční tabulce.
\item[\emph{Proof Table Hits} a \emph{Proof Table Misses}] Podobně jako 
  příslušné hodnoty pro transpoziční tabulku.
\item[\emph{History Table Size}] Počet záznamů v \emph{history table}.
\item[\emph{Total Killer Count}] Celkový počet záznamů v tabulce killerů.
\end{description}

\section{Prohledávací algoritmus}
Program funguje postupným prohledáváním všech možných pozic vzniklých ze zadané
počáteční pozice. Pozice jsou uloženy ve vyhledávacím stromu, kde každý vrchol 
představuje jednu možnou pozici a jeho potomci jsou vrcholy vzniklé provedením
nějakého platného kroku. Celý tento strom představuje \emph{stavový prostor}
hry.

Algoritmus \emph{Proof-Number Search} je \emph{best-first} algoritmus na
prohledávání stavového prostoru. Pracuje postupným rozšiřováním stavového 
prostoru dokud nedojde k výsledku. Přitom v každé iteraci vždy najde jeden list 
stromu a ten \emph{expanduje} -- vygeneruje všechny možné platné kroky z listu 
a každý jako nový potomek listu připojí ke stromu.

Jeden vrchol stromu se nazývá \emph{dokázaný}, pokud má útočník zaručené 
vítězství, dostane-li se hra do tohoto vrcholu. Pokud tento vrchol zaručuje 
vítězství obránce, pak je vrchol \emph{vyvrácený}. Cílem algoritmu je dokázat či
vyvrátit kořenový vrchol.

Kvůli velikosti celého stavového prostoru hry je vhodné snažit se prozkoumávat
nejdříve ty části stavového prostoru, u kterých je nejvyšší šance, že v nich 
algoritmus nalezne řešení celého problému -- tedy že buď najde způsob, jakým může
útočník zvítězit nebo prokáže, že tento hráč nikdy zvítězit nemůže. K tomuto
účelu má každý vrchol přiřazena dvě čísla: \emph{PN} (\emph{proof number}) a 
\emph{DN} (\emph{disproof number}). Číslo \emph{PN} udává spodní odhad počtu
vrcholů, které je potřeba dokázat na dokázání tohoto vrcholu; \emph{DN} zase
udává spodní odhad na počet vrcholů potřebných na vyvrácení tohoto vrcholu.

\subsection{Proof-number a disproof-number}
Každý vrchol má dále přiřazen typ -- ten je buď \emph{AND} nebo \emph{OR}. Typ
\emph{OR} znamená, že z tohoto vrcholu táhne útočník. Typ \emph{AND} zase, že 
táhne obránce. Kořen je vždy typu \emph{OR}.

Pro dokázání vrcholu typu \emph{OR} stačí dokázat jednoho potomka -- útočník
vybírá svůj tah a může si tedy vybrat ten nejlepší. Naopak pro vyvrácení takového 
vrcholu je nutné vyvrátit úplně všechny potomky -- neboli dokázat, že útočník 
vždy prohraje, ať zvolí jakýkoliv tah.

U vrcholu typu \emph{AND} je situace opačná. Dokázat takový vrchol znamená, že
ať soupeř zvolí jakýkoliv tah, útočník vyhrává. Je tedy nutné dokázat všechny 
potomky. Pro jeho vyvrácení stačí vyvrátit jediného potomka.

Pro vrchol $v$ typu \emph{OR} jsou tedy čísla $v.PN$ a $v.DN$ určena následovně:
\begin{align*}
  v.PN &:= \min_{u \text{ je potomek $v$}} u.PN \\
  v.DN &:= \sum_{u \text{ je potomek $v$}} u.DN
\end{align*}

Pro vrchol typu \emph{AND} se tyto hodnoty počítají analogicky.

Ve formulích výše se za minimum z prázdné množiny bere $+\infty$, za součet přes
prázdnou množinu zase $0$. Tím se implicitně detekují situace, kde některý z
hráčů prohrává tím, že nemá žádný možný krok.

\subsection{Procházení stromu}
Při sestupu stromem od kořene k listu se program snaží najít takový list, na
jehož dokázání či vyvrácení bude třeba co nejméně práce. Vybírání listu probíhá 
následovně:

\begin{figure}[H]
{\bf procedure} $\text{findLeaf}(v)$
\begin{algorithmic}
  \IF{$\text{leaf}(v)$}
    \STATE {\bf return} v
  \ELSIF{$\text{type}(v) = OR$}
    \STATE $u \gets \mathrm{arg\, min}_{c \in \mathrm{children}(v)} c.PN$
    \STATE {\bf return} $\text{findLeaf}(u)$
  \ELSE
    \STATE $u \gets \mathrm{arg\, min}_{c \in \mathrm{children}(v)} c.DN$
    \STATE {\bf return} $\text{findLeaf}(u)$
  \ENDIF
\end{algorithmic}
\end{figure}

\subsection{Rozvíjení listu}
Při rozvíjení listu program vygeneruje všechny možné kroky z dané pozice. Nové 
kroky coby vrcholy se pak připojí k rozvíjenému vrcholu. Je dobré uvědomit si, 
že pokud daný hráč nemá žádný platný krok, který by mohl z rozvíjeného stavu
učinit, k listu se nepřipojí žádní noví potomci.

Hra Arimaa má dvě pravidla, která znamenají, že ne úplně každý možný krok je 
platný:
\begin{enumerate}
\item Po skončení celého tahu jednoho hráče musí být rozestavění figurek na
  hrací ploše jiné než na začátku tahu. Tedy hráč nesmí udělat ,,prázdný`` tah.
\item Pokud po skončení celého tahu hráče vznikne potřetí v celé historii
  jedné hry stejná pozice, ze které má táhnout stejný hráč, pak hráč, který 
  právě skončil tah prohrává. Tzv. pravidlo \emph{third-time repetition}.
\end{enumerate}

Tato dvě pravidla mimo jiné zajišťují konečnost každé hry. Pro vyhledávací
algoritmus je konečnost hry podstatná, protože jinak by mohl ve smyčce rozvíjet 
stále stejné pozice, čímž by neobjevoval žádné nové stavy, ale akorát zbytečně
spotřebovával čas a paměť.

Pro zohlednění tohoto pravidla si algoritmus při sestupu z kořene k listu navíc
vytváří \emph{historii} -- seznam dvojic $(\text{pozice}, \text{hráč})$ pro 
každý první krok každého tahu. Před připojením listu představující ukončení tahu
aktuálního hráče se nejprve zkontroluje, zda tím nevznikne nějaká pozice, která
už je v historii. Pokud se zjistí porušení některého ze dvou výše uvedených 
pravidel, tak se tato pozice připojí jako dokázaná či vyvrácená v závislosti
na tom, který hráč se dopustil porušení pravidla.

Rozvíjení listu pak vypadá následovně:
\begin{figure}[H]
{\bf procedure} $\text{expand}(\text{leaf}, \text{history})$
\begin{algorithmic}
\FORALL{$s \gets \text{possibleSteps}(\text{leaf})$}
  \IF{$\text{type}(s) = \text{type}(\text{leaf})$}
    \STATE $\text{attach}(\text{leaf}, \text{Vertex}(s, PN=1, DN=1))$
  \ELSE
    \STATE $pos \gets \text{position}(\text{leaf}, s)$
    \STATE $pl \gets \text{opponent}(\text{leaf})$
    \STATE $rep \gets \left|\left\{ (a, b) \in \text{history} \;:\; 
                                    a = pos, b = pl \right\}\right|$
    \IF{$rep \ge 3 \lor (pos, pl) = \text{last}(\text{history})$}
      \STATE $\text{attach}(\text{leaf}, \text{Vertex}(s, \text{lose}))$
    \ELSE
      \STATE $\text{attach}(\text{leaf}, \text{Vertex}(s, PN=1, DN=1))$
    \ENDIF
  \ENDIF
\ENDFOR
\end{algorithmic}
\end{figure}

\subsection{Heuristiky}
\label{sec:heuristics}
Naivní průchod, jak byl popsán výše, je poměrně málo efektivní. Je výhodné
sdílet některé údaje mezi jednotlivými větvemi stromu, aby se některé větve
nemusely vůbec rozvíjet. K tomu je v programu zahrnuto několik heuristik.

\subsubsection{Transpoziční tabulka}
\label{sec:heur-tt}
Při prohledávání stavového prostoru se dost často stává, že program dojde jinou
cestou do pozice, kterou již dříve prozkoumal. Aby se zabránilo opakovanému
prozkoumávání podstromů, které odpovídají již prozkoumané části hry, ukládají
se hodnoty PN a DN do \emph{transpoziční tabulky}. Transpoziční tabulka je
asociativní pole indexované \emph{hashem vrcholu}. Hash vrcholu je nějaké
64bitové číslo s tím, že dvě stejné pozice (stejné rozestavení figurek a stejný
hráč na tahu) mají stejný hash.

Protože všech možných hashů je \(2^{64}\), jako index do transpoziční tabulky
se bere hash modulo \(N\), kde \(N\) je uživatelem nastavitelná hodnota 
odpovídající maximálnímu možnému počtu záznamů v transpoziční tabulce.

Při vytváření nových vrcholů se program nejprve podívá do transpoziční tabulky
a, najde-li tam příslušný záznam, použije hodnoty z ní.

Při použití transpoziční tabulky jsou dva problémy: První je možnost kolizí --
jak v hashi samotném, tak ve skutečném indexu, tedy \(H \bmod N\). Druhý je
problém \emph{graph history interaction} -- díky pravidlu third-time repetition
je možné, že podstrom, který se právě rozvíjí, může vypadat jinak než ten, ze
kterého se do transpoziční tabulky uložily hodnoty. To by mohlo vést k 
nesprávným výsledkům, pokud by se z transpoziční tabulky braly informace o
dokázaných či vyvrácených vrcholech (tedy záznamy s PN či DN rovnými nule). 

Možné řešení by bylo ukládat do transpoziční tabulky informace o historii
každého uloženého vrcholu. To by však stálo prostor i výpočetní čas pro 
ověřování historií. V programu jsou tyto problémy vyřešeny tak, že se do
transpoziční tabulky neukládají dokázané ani vyvrácené vrcholy. Tím se při
vyzvednutí špatných hodnot může stát, že část výpočtu bude neefektivní, ale 
zabraňuje to vydání špatných výsledků.

Pro řešení kolizí (tedy situací, kde v tabulce je záznam s hešem \(H_1\) a má se
do ní uložit záznam s hešem \(H_2\), přičemž \(H_1 \equiv H_2 \pmod N\)) se ke
každému záznamu v transpoziční tabulce se ukládá i původní heš, aby bylo možné 
tyto kolize detekovat. V případě kolize se preferuje záznam s větší hloubkou (za 
hloubku se považuje ply záznamu, tedy počet tahů od kořene).

\subsubsection{Důkazová tabulka}
\label{sec:heur-pt}
Informace o dokázaných či vyvrácených vrcholech jsou i tak poměrně důležité a
můžou algoritmu pomoct ve výpočtu. Proto jsem zavedl navíc ještě \emph{důkazovou
tabulku (proof table)}. Ta funguje v principu stejně jako transpoziční tabulka,
s několika rozdíly:\begin{enumerate}
\item Ukládají se do ní \emph{pouze} dokázané či vyvrácené vrcholy (tedy 
  informace v transpoziční a důkazové tabulce jsou navzájem disjunktní).
\item Ke každému záznamu se navíc ukládá historie vrcholu, ze kterého záznam
  pochází.
\end{enumerate}

\emph{Historie} v důkazové tabulce je seznam hešů všech těch pozic na cestě od 
kořene k příslušnému vrcholu, ze kterých začíná svůj tah jeden z hráčů. (Neboli
těch vrcholů, které mají opačný typ než jejich rodič, případně kořenového
vrcholu.)

Při vyzvedávání záznamu z důkazové tabulky se porovnává uložená historie s 
historií vrcholu, ke kterému se hledají hodnoty PN, DN. Pro každý hash z obou
historií se spočítá, kolikrát se celkem v obou historiích vyskytuje. Pokud se
zjistí, že některý hash se vyskytuje aspoň třikrát, je záznam zamítnut a
považuje se, že v důkazové tabulce vůbec nebyl nalezen.

Důkazová tabulka je oddělená od transpoziční kvůli větší spotřebě paměti a
časově náročnějšímu vyhledávání záznamů. Takto se cena za ukládání historie
platí pouze u záznamů, u kterých je to nezbytně nutné.

\subsubsection{Killery}
\label{sec:heur-killers}
Další informace, kterou je výhodné sdílet mezi větvemi stromu, jsou vrcholy,
které někdy v průběhu prohledování způsobily \emph{cutoff} -- tedy to, že jejich
sousedi se již nemusely prohledávat. (Pro OR vrchol to je dokázaný potomek, pro
AND vrchol zase potomek vyvrácený.) Dá se očekávat, že tento vrchol způsobí
cutoff i ve vedlejší větvi -- například pokud se tyto větve liší jenom pozicí
některé málo významné figurky.

Pro každou hladinu stromu (počet vrcholů od kořene) se udržuje pro každý typ
seznam \(K\) kroků, které v minulosti způsobily cutoff. Hodnota \(K\) je 
nastavitelná uživatelem -- protože pro každý typ se udržuje jeden seznam,
celkem je pro každou hladinu udržováno až \(2K\) záznamů.

Před samotným rozvinutím listu se nejdříve v tabulce killerů algoritmus pokusí
najít krok, který je z listu platný podle pravidel hry, a pokud se nějaký najde,
připojí se jako nový list, který se vyhodnotí. Pokud tento vrchol způsobí cutoff,
není třeba jeho sourozence vůbec vytvářet. Pokud cutoff nezpůsobil, zkouší se
rekurzivně killery o hladinu níž. Protože cílem je najít cutoff, zkouší se jenom
kroky jednoho hráče -- rekurze má tedy nejvíš čtyři hladiny, kde na každé se
zkouší nejvýš \(K\) kroků.

Popsaný proces se nazývá \emph{simulace}. Je to snaha použít cestu k důkazu,
která byla úspěšná ve vedlejší větvi bez rozvíjení celého podstromu.

Pokud simulace selže -- tedy se nepovede najít cutoff -- pak se přejde k
normálnímu rozvinutí listu, ovšem s tím, že vrcholy, které jsou pro danou
hladinu vedeny jako killery, se dají na první místo v seznamu potomků.

\subsubsection{History table}
\label{sec:heur-ht}
Druhý obecnější způsob na sdílení cutoffů mezi větvemi je \emph{history table}.
Je to asociativní pole indexované krokem. Pokud vrchol v hladině \(h\) způsobí
cutoff, do history table se k záznamu pro krok vedoucí do toho vrcholu přičte
hodnota \(h^2\).

Po rozvinutí listu se jeho noví potomci uspořádají podle hodnot z history table
-- čím vyšší hodnota přísluší nějakému vrcholu, tím blíž je začátku seznamu
následníků. Při řazení následníků se vždy na první místa dávají killery a 
teprve ostatní místa se řadí podle history table.

\subsection{Depth-First varianta}
\label{sec:dfpns}
Pro snížení paměťové náročnosti algoritmu jsem implementoval i depth-first
variantu. Pro každý vrchol se navíc udržují jeho \emph{limity} -- dvojice 
čísel PNt a DNt. Prohledávání podstromu pak pokračuje tak dlouho, dokud tento
podstrom není buď dokázaný či vyvrácený nebo dokud PN nepřesáhne limit PNt či
DN nepřesáhne DNt. Kořenu stromu má oba limity nastaveny na \(+\infty\).

Buďte \(p, v, u\) vrcholy s tím, že \(u, v\) jsou potomci \(p\). Pokud \(p\) je 
typu OR, \(v\) je nejlepší následník a \(u\) je druhý nejlepší následník, pak
limity pro \(v\) jsou dány následovně:
\begin{align*}
  v.pnt &:= \min \{ p.pnt, \lceil u.pn (1 + \epsilon) \rceil \} \\
  v.dnt &:= p.dnt - p.dn + v.dn
\end{align*}

\(\epsilon\) je v současné implementaci rovno \(0.1\).

Algoritmus rozvíjí podstrom dokud jsou PN, DN kořene podstromu menší než
příslušné limity. Při překročení limitu se s rozvíjením podstromu končí a 
algoritmus se vrací po rodičích směrem ke kořeni, dokud nenajde takový vrchol,
jehož PN, DN vyhovují limitům (aspoň kořen stromu tuto podmínku vždy splňuje).

Jako optimalizace se při vracení směrem ke kořeni neodřezávají potomci každého
vrcholu, ale jen těch vrcholů, jejichž typ je rozdílný od typu potomka, který
byl dosud na prohledávané cestě. Neboli odřezávání probíhá po tazích místo po
krocích.

V psedudokódu jedna iterace algoritmu \emph{dfpns} vypadá takto:
\begin{figure}[H]
\begin{algorithmic}
  \WHILE{\(\text{proved}(v) \lor \text{over\_limits}(v)\)}
    \STATE \COMMENT{Vzestup po rodičích k vrcholu v rámci limitů}
    \IF{\(v.\text{type} \ne \text{parent}(v).\text{type}\)}
      \STATE \(\text{remove}(v)\)
    \ENDIF
    \STATE \(v \gets \text{parent}(v)\)
  \ENDWHILE
  
  \COMMENT{\(v\) nyní je vrchol splňující limity}
  \WHILE{\(\neg \text{leaf}(v)\)}
    \STATE \(v \gets \text{best\_successor}(v)\)
  \ENDWHILE
  
  \STATE \(\text{expand}(v)\)
  \STATE \(\text{update\_path}(v, \text{parent}(v), 
                               \text{parent}(\text{parent}(v)),
                               \ldots,
                               \text{root})\)
\end{algorithmic}
\end{figure}

\subsection{Garbage collector}
Algoritmus má přese všechny heuristiky poměrně velkou paměťovou náročnost. Jako
ad-hoc řešení jsem zavedl \emph{garbage collector}. Jeho parametry jsou dvě
čísla, obě udávající počet vrcholů: \emph{GC high} a \emph{GC low}.

Pokud je GC zaplý a počet vrcholů překročí GC high, prohledávání se
přeruší a ze stromu se začnou ořezávat vrcholy tak dlouho, dokud velikost
stromu neklesne pod GC low nebo již není možné odstranit žádný vrchol.

GC funguje rekurzivně: Nejprve se na úrovni pod kořenem odstraní následníci 
všech vyvrácených vrcholů. Pokud je stále velikost celého stromu nad hodnotou 
GC low, vybere se z této úrovně nejhorší vrchol, který není list, a kolektor do 
něj sestoupí. Tam se opět odstraní potomci vyvrácených (či dokázaných) vrcholů 
a pokud je stále velikost stromu nad GC low, pokračuje se rekurzivně.

Jakmile již není možné v rekurzi pokračovat, protože všichni následníci jsou
listy, odstraní se všichni tito následníci a kolektor se vynoří o jednu hladinu
z rekurze. Tam se opět vybere nejhorší následník, který není list (pokud takový
existuje) a pokračuje se stejným způsobem.

V pseudokódu:
\begin{figure}[H]
{\bf procedure} \(\text{GC}(v)\):
\begin{algorithmic}
  \FORALL{\(c \in \text{children}(v)\)}
    \IF{\(\left( v.\text{type} = \text{OR} \land c.\text{dn} = 0 \right)
        \lor \left( v.\text{type} = \text{AND} \land c.\text{pn} = 0 \right)\)}
      \STATE \(\text{remove\_children\_of}(c)\)
    \ENDIF
  \ENDFOR
  
  \WHILE{\(\text{tree\_size} > \text{gc\_low}\)}
    \STATE \(w \gets \text{ worst non-leaf successor of } v\)
    \IF{\(w \ne \text{NULL}\)}
      \STATE \(\text{GC}(w)\)
    \ELSE
      \STATE \(\text{remove\_children\_of}(v)\)
    \ENDIF
  \ENDWHILE
\end{algorithmic}
\end{figure}

Tímto způsobem se algoritmus GC snaží odstraňovat vrcholy postupně od nejhoršího 
a nejhlubšího, aby se jako první odstranily ,,málo potřebné`` vrcholy, či 
takové, ve kterých není zakořeněný příliš hluboký podstrom, který by se pak
musel případně počítat odznova.

\section{Překlad programu}
\label{sec:compiling}

\subsection{Prerekvizity}
Pro překlad je třeba mít kompletní zdrojový kód programu. Ten lze obstarat přes
systém git příkazem
\begin{center}\texttt{git clone git://github.com/Oxyd/APNS.git}\end{center} či
stažením zdrojového archivu ve formátu
\texttt{.zip} či \texttt{.tar.gz} z \url{https://github.com/Oxyd/APNS}.

Dále je třeba mít v systému instalovány následující knihovny a
nástroje\begin{description}
\item[SCons] Používaný nástroj pro sestavení programu. Instalátor pro Windows je
  možné získat na
  \url{http://scons.org/}. Na distribucích GNU/Linuxu stačí zpravidla instalovat 
  příslušný balíček od distributora.
\item[Boost] Kolekce knihoven, z nichž některé program používá. Je třeba mít
  verzi aspoň 1.44. Stáhnout lze z
  \url{http://boost.org/}, kde jsou i informace pro sestavení kolekce. Na 
  systému GNU/Linux opět zpravidla distributor dodává předpřipravený balíček.
\item[Google Test] Není nezbytně nutný pro překlad; používá se pouze pro
  jednotkové testy. Pro Windows se stačí obrátit na stránky projektu na
  \url{http://code.google.com/p/googletest/}. Je zapotřebí verze aspoň 1.5.
\end{description}

Je třeba dále mít k dispozici překladač jazyka C++. Sestavovací skript
automaticky použije překladač z \emph{Microsoft Visual Studia} na systému 
Windows -- je třeba mít verzi 10. (Měla by stačit i verze 9 a 8, ale to není 
otestované.) Na GNU/Linuxu se pro překlad používá překladač \emph{GCC} -- je 
ovšem třeba mít překladač verze aspoň 4.5.

Program je možné na GNU/Linuxu přeložit i překladačem Clang -- sestavovacímu
skriptu stačí předat parametr \texttt{toolchain=clang}.

Na Windows není podporován překlad jiným překladačem než z \emph{Visual Studia}.
Důvodem je špatná podpora překládání modulů jazyka Python jiným překladačem než
tímto.

\subsection{Konfigurace pro překlad}
Různé překladové parametry -- zejména cesty ke knihovnám -- je možné nastavit
v souboru \texttt{config/\textit{platforma}.py}. Při použití systémového 
překladače na systému GNU/Linux by žádná konfigurace neměla být potřeba. Na 
systému Windows bude třeba v \texttt{config/windows.py} správně nastavit cesty k
hlavičkovým souborům a statickým knihovnám.

\subsection{Překlad}
Překlad lze spustit příkazem \texttt{scons} v kořenovém adresáři se zdrojovým
kódem programu. Pokud překlad proběhne úspěšně, není třeba žádných dalších 
kroků; program je spustitelný tak, jak bylo popsáno v části \ref{sec:running}.

Pro účely vývoje programu může být vhodné přeložit program tak, aby obsahoval
ladicí informace. V tom případě je třeba překlad spustit příkazem 
\texttt{scons debug=1}. Navíc je možné specifikovat cíl \texttt{tests}, který 
přeloží a provede jednotkové testy -- tedy použít příkaz \texttt{scons tests}. 
Jednotlivé jednotkové testy je možné vyvolat cílem \emph{jméno\_testu}, kde 
\emph{jméno\_testu} odpovídá jménu souboru v adresáři 
\texttt{apns\_module/tests} bez koncovky \texttt{.cpp}. Například tedy pro 
spuštění jednotkových testů ze souboru 
\texttt{apns\_module/tests/board\_test.cpp} se dá použít příkaz 
\texttt{scons board\_test}.

Při překladu na systému Windows je též možné specifikovat parametr
\texttt{bits=32} nebo \texttt{bits=64}, který určuje, zda má výsledný program 
být přeložen pro 32bitovou nebo 64bitovou verzi systému. Kromě toho existuje cíl
\texttt{distrib}, který vytvoří \texttt{zip} soubor s binární distribucí
programu.

\section{Vstupní a výstupní data}
\subsection{Pozice}
Vstupní pozice je textový soubor ve formátu popsaném na adrese
\url{http://arimaa.com/arimaa/learn/notation.html} v části 
\emph{Notation for Recording Arimaa Positions}. Program ale ignoruje číslo tahu 
a předešlé kroky. Konkrétní příklady pozic jsou v adresáři 
\texttt{example-positions}.

\subsection{Strom}
Stavový strom je ukládán v textovém formátu. Struktura souboru je
\begin{itemize}
\item Na první řádce je počáteční pozice v kompaktním formátu: Seznam figurek
  na hrací ploše, které jsou oddělené mezerami. Jedna figurka je popsána 
  trojznakovým řetězcem, kde první znak určuje řádek ('1' až '8'), druhý značí
  sloupec ('a' až 'h') a třetí znak typ figurky podle oficiálního značení: 'E'
  značí slona, 'M' velblouda, 'H' koně, 'D' psa, 'C' kočku a 'R' zajíce. Typ je 
  značen velkým písmenem, pokud je to figurka zlatého hráče -- figurky 
  stříbrného hráče jsou značeny písmeny malými. Například: \begin{center}
  \texttt{5cC 6bE 6cd 6dR 6ee 7cr}\end{center}
\item Na druhé řádce je hráč, který je první na tahu jako řetězec buď 
  \texttt{gold} nebo \texttt{silver}.
\item Následuje seznam vrcholů stromu. Jeden vrchol je na jedné řádce. Každý
  vrchol je popsán následovně: \begin{center}
    \texttt{\textit{tah} : \textit{typ} \textit{kroky} \textit{PN} \textit{DN} 
      \textit{děti}}
    \end{center}
  Kde jednotlivé položky jsou:\begin{description}
  \item[\textit{tah}] Textový popis tahu vedoucího do této pozice z
    rodičovského vrcholu. Používá se oficiální notace pro zápis tahů hry Arimaa. 
    Kořen má místo tahu řetězec \texttt{root}.
  \item[\textit{typ}] Řetězec \texttt{or} nebo řetězec \texttt{and}. Udává typ 
    vrcholu.
  \item[\textit{kroky}] Číslo v dekadickém zápisu udávající počet kroků, které 
    ještě může hráč učinit v rámci svého tahu.
  \item[\textit{PN}] Hodnota \emph{proof number} pro tento vrchol v dekadickém 
    zápisu nebo řetězec ,,\textit{infty}`` značící nekonečno.
  \item[\textit{DN}] Hodnota \emph{disproof number} pro tento vrchol v 
    dekadickém zápisu nebo řetězec ,,\textit{infty}`` značící nekonečno.
  \item[\textit{děti}] Počet potomků tohoto vrcholu v dekadickém zápisu.
  \end{description}
\item Vrcholy jsou ve stromu uloženy tak, že první je kořen a po každém vrcholu 
  následují jeho potomci. Strom je tedy v souboru uložen v pořadí pre-order 
  průchodu do hloubky.
\end{itemize}

\section{Implementace}
\subsection{Struktura programu}
Program sestává ze dvou částí: Nativního modulu, který implementuje algoritmus
samotný a pomocné datové typy a algoritmy pro reprezentaci hry; a uživatelského 
rozhraní, které je implementované v jazyku Python.

Algoritmus je v nativním modulu kvůli běhové rychlosti nativního kódu a kvůli
tomu, že umožňuje detailnější správu paměti. Naproti tomu uživatelské rozhraní 
není příliš výpočetně náročné, tedy není třeba mít ho v nativním kódu. Jazyk
Python navíc obsahuje spoustu předpřipravených knihoven, které velmi usnadňují
vývoj grafického (i dávkového) uživatelského rozhraní.

Implementace celého programu v jazyku C++ by vedlo k přílišnému zesložitění kódu
pro interakci s uživatelem. To by nebyl takový problém, pokud by program žádné 
grafické rozhraní neměl -- připadá mi ale, že vizualizace stromu a jednotlivých
pozic umožňuje uživateli vytvořit si lepší představu o funkci algoritmu.

Rovněž by bylo možné implementovat celý projekt jako několik programů: Samotný
výpočetní program, který by neměl žádné (nebo jen minimální) uživatelské 
rozhraní a pak další programy, které by jeho výstup nějak zpracovávaly a 
vizualizovaly. To by znamenalo, že program nebude během výpočtu platit za 
nezbytnou existenci interpreteru jazyka Python v paměti jak je tomu teď. Na 
druhou stranu by však nemohl program poskytovat uživateli průběné informace 
během výpočtu -- leda by současně běžel druhý program s uživatelským rozhraním 
a oba programy by spolu nějakým způsobem komunikovaly -- to by ovšem 
představovalo stejný overhead jako nyní a navíc overhead na komunikaci mezi 
procesy. Dále by se některé datové struktury musely duplikovat ve více 
programech -- počítací program by měl vlastní reprezentaci stromu v paměti, ke 
které by však uživatelské rozhraní nemělo přístup. Program s uživatelským 
rozhraním by pak musel mít vlastní reprezentaci stromu.

Výpočetní část programu je tedy implementována jako nativní modul jazyka Python.
Vnitřní datové struktury -- jako strom samotný či hrací plocha a reprezentace 
kroků -- jsou přímo přístupné kódu v Pythonu a nemusí se duplikovat. To také
znamená, že je teoreticky možné implementovat část algoritmu v Pythonu --
například různé heuristické funkce. Tím sice jistě vznikne netriviální overhead, 
ale umožní to snadno testovat experimentální vylepšení algoritmu -- pokud se
vylepšení prokáže jako rozumné, nebude problém ho přepsat do nativního modulu.

Rovněž je možné spustit interpret Pythonu v interaktivním režimu, v něm načíst
výpočetní modul a interaktivně spouštět části algoritmu či procházet vzniklé 
datové struktury. To se mi prokázalo jako poměrně praktický nástroj při ladění
programu.

\subsection{Reprezentace dat}
\subsubsection{Hrací plocha}
Hrací plocha je reprezentována jednoduše jako 64-prvkové pole, kde každý prvek
je tvořen jedním znakem, který reprezentuje figurku na příslušné pozici.

\subsubsection{Kroky}
Krok -- změna hrací plochy z jednoho platného stavu do druhého platného stavu --
je reprezentován jako řetězec ve stejném formátu v jakém jsou kroky popisovány v 
oficiálních zápisech hry. Program vytváří poměrně mnoho kroků a prakticky se mi 
ukázalo vhodné zvolit tuto poměrně kompaktní reprezentaci oproti přímějším 
variantám.

Krok samotný navíc nevyžaduje, aby k němu byla v paměti uložena odpovídající
hrací plocha. V každém vrcholu stromu je proto uložen jenom krok, kterým vznikne 
patřičná hrací plocha z rodičovského vrcholu, namísto aby tam byla uložena celá
hrací plocha.

\subsubsection{Vrchol stromu}
Kromě nezbytných informací o kroku, hodnotách \emph{PN} a \emph{DN} je v každém
vrcholu třeba udržovat informaci o jeho potomcích.

Pro reprezentaci potomků jsem zvolil pole ukazatelů na následníky. To poněkud
usnadňuje implementaci, jelikož je možné v programu pracovat s ukazateli na
některé vrcholy bez toho, že hrozilo přesunutí těchto vrcholů zatímco jiná část
programu drží ukazatel na některý z nich.

Nevýhodou je, že pro přístup k následníkovi vrcholu je potřeba prvně přístup do
paměti k seznamu ukazatelů a následně druhý přístup k vrcholu samotnému. 

\section{Závěrem}
Můj původní záměr byl reprezentovat stavový prostor hry jako orientovaný
acyklický graf (DAG) -- místo používání pouze hodnot \emph{PN}, \emph{DN} z 
transpoziční tabulky by se k rozvíjenému listu připojil přímo již existující 
vrchol. To by znamenalo zefektivnění výpočtu, protože většina duplicitních pozic 
by nemusela být ve stromu vícekrát.

Zajištění acykličnosti grafu se však prokázalo jako netriviální ale velmi
podstatné. Cyklus v grafu neznamená jenom možnost zacyklení celého algoritmu při 
výběru nejlepšího listu, ale též to nereprezentuje správně stavový prostor hry 
-- pravidla explicitně znemožňují opakování stále těch samých pozic dokola 
(pravidlo \emph{third-time repetition}).

Nakonec jsem od tohoto záměru upustil a rozhodl se pro jednodušší, i když méně
efektivní reprezentaci obyčejným stromem. Další možností by bylo DAGem
reprezentovat pouze tah jednoho hráče -- tím by nemohly vzniknout problémy s
opakování stále stejných pozic (protože pravidla o opakování se týkají jenom
pozic po ukončení tahu hráče) a vedlo by to k aspoň nějaké redukci počtu 
vrcholů.
\end{document}
% vim:textwidth=120
